\section{Design}

To achieve fast analytics of massive datasets spread over the wide area, we have implemented several key components as part of Meteor, as described below.

\subsection{Location-Aware Map}

\begin{figure}[!ht]
\centering\includegraphics[width=0.5\columnwidth]{figs/lam.pdf}
\vspace{-1.2em}
\caption{Scheduler keeps separate task queues for each cluster/host in location-aware map.}
\label{fig:lam}
\vspace{.7em}
\end{figure}

The system should reduce reliance on network bandwidth since it's a scarce resource. To prevent backhauling
all the data from every location to a centralized location at the beginning of the job, input data should remain in place. To prevent map inputs shuttled over the wire, we implement Location-Aware Map (LAM). LAM scheduling achieves data locality by essentially forbidding a node executing a map task over a data partition residing in a remote cluster. Conceptually LAM scheduling is shown in \ref{fig:lam}. Instead of a FIFO queue of tasks, the scheduler has a separate task queue for each cluster $n$. Therefore a node in cluster $n$ executes a map task that
belonged to its own queue. We provide an API call in Spark to let the programmer attach each data source, i.e. RDD, to a set of nodes. The scheduler puts each map task for partitions for this data source in the appropriate queue. It is interesting to note that Location-Aware Map is equivalent to Delay-Scheduling \cite{delay-scheduling} with infinite delay.

\subsection{Bandwidth-Aware Work Stealing}
