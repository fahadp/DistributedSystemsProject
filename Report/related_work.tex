\section{Related Work}

\paragraph{Data-centric programming models for parallel in-memory applications.}
Spark, and Spark-streaming \cite{spark} are recently developed systems for big-data processing, with Spark-streaming an extension to Spark for streaming applications. The primary innovation of these systems is to keep intermediate data for MapReduce jobs in memory, providing quicker response times. Although these systems provide real-time data analytics with fault tolerance, they assume stable, high bandwidth shared storage, such as HDFS. Meteor, on the other hand, provides fast analytics over multiple datacenters connected by a WAN. We will primarily use Spark as the base for Meteor.  

Piccolo \cite{piccolo} exposes mutable state to allow the user to perform in-memory computation via a key-value table interface. Piccolo lets applications specify their locality policies to exploit the locality of shared state. When loading data from distributed files, piccolo chooses an assignment that minimizes inter-rack transfer. Additionally, Piccolo implements a greedy heuristic-based work-stealing approach to load-balancing. Piccolo, as opposed to Spark, offers an interface based on fine-grained updates to a mutable state. With such interface, providing fault tolerance can become expensive for data intensive workloads, as they require copying large amounts of data over the cluster network. 

\paragraph{Approximate Queries.}
BlinkDB \cite{blinkdb} is a massively parallel approximate query engine for running interactive SQL queries on large volumes of data. It allows users to trade-off query accuracy for response time by running queries on data samples. The results are presented to the results to users with meaningful error bars. BlinkDB relies on an optimization framework that maintains multi-dimensional stratified samples from original data over time in order to perform dynamic selection. BlinkDB therefore only applies to centralized databases where dynamic stratification is made possible. It also is restricted to aggregate queries, instead of a broader set of MapReduce jobs.

HOP \cite{hop} is an online version of the MapReduce architecture that supports online aggregation, which allows users to see “early returns” from a job as it is being computed. Online aggregation is a technique that provides initial estimates of results several orders of magnitude faster than the final results, by pre-emptively applying the reduce function to the data that a reduce task has received so far at a given point in time. Meteor borrows the idea of  invoking the reduce function on partial map tasks outputs, but applies it in order to reduce the amount of data exchange between workers – not in order to perform interactive analysis.

There’s has been a great deal of theory research on streaming algorithms for distributed queries. These algorithms try to limit message-passing by introducing tolerance for inaccurate final results. For example, distributed top-k monitoring \cite{topk} and approximate quantiles \cite{approxquant}. We intend to use many ideas contained in this paper as we figure out the operators that Meteor should support. 

\paragraph{Resource-constrained computing.}
There’s has also been a great deal of research in sensor networks on resource-constrained computing, such as \cite{tag}. Although we intend to study and use some of these techniques, our focus is primarily on efficient use of network bandwidth rather than the traditional sensor network focuses such as limited power and/or processing.  
